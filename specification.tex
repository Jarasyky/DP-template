%CZ parametry práce
\newcommand{\nazevCZ}{\textit{Název Práce}}	% český název práce
\newcommand{\skolaCZ}{Západočeská univerzita v Plzni}	% název školy
\newcommand{\ZskolaCZ}{ZČU}	% zkratka školy
\newcommand{\fakultaCZ}{Strojní}	% fakulta
\newcommand{\ZfakultaCZ}{FST}	% zkratka fakulty
\newcommand{\programCZ}{Stavba energetických strojů a~zařízení}  % název studijního programu
\newcommand{\programN}{\textit{B0715A270013}}	% číslo studijního programu
\newcommand{\oborCZ}{Stavba energetických strojů a~zařízení}	% název oboru
\newcommand{\katedraCZ}{Katedra energetických strojů a zařízení}	% název zadávající katedry
\newcommand{\ZkatedraCZ}{KKE}	% zkratka katedry
\newcommand{\pracovisteCZ}{\ZskolaCZ--\ZfakultaCZ--\ZkatedraCZ}	% název pracoviště
\newcommand{\oborN}{N/A}	% číslo oboru
\newcommand{\druhCZ}{\textit{Bakalářská práce/Diplomová práce}}	% Bakalářská/Diplomová práce
\newcommand{\klicovaCZ}{zvuk, akustika, turbínová spojka, rezonance, Helmholtzův rezonátor, měření zvuku, akustická hladina, frekvenční pásmo, váhový filtr, kondenzátorový mikrofon, akustická kamera}	% klíčová slova
\newcommand{\abstrCZ}{Bakalářská práce se zaměřuje na téma hluku turbínových spojek. Úvodem práce poskytuje teoretické poznatky o fyzikálních vlastnostech zvuku. Dále je součástí práce popis experimentu a vyhodnocení naměřených dat. V neposlední řadě se zabývá technickým řešením samotného problému a navrhuje konstrukční úpravu, která by mohla problematiku řešit. Tuto úpravu však zatím není v praxi možné využít kvůli nejistotě míry účinnosti.}	% abstrakt

%EN parametry práce
\newcommand{\nazevEN}{Noise of turbine clutches} % název práce
\newcommand{\skolaEN}{University of West Bohemia}	% název školy
\newcommand{\fakultaEN}{mechanical engineering}	% název fakulty
\newcommand{\programEN}{Mechanical Engineering}  % název katedry
\newcommand{\oborEN}{Design of Power Machines and Equipment} 		% název oboru
\newcommand{\druhEN}{Bachelor thesis}	% druh práce
\newcommand{\klicovaEN}{sound, acoustics, turbine coupling, resonance, Helmholtz resonator, sound measurement, sound level, frequency band, weighting filter, condenser microphone, acoustic camera}	% klíčová slova
\newcommand{\abstrEN}{\foreignlanguage{english}{The bachelor thesis focuses on the topic of turbine coupling noise. The introduction of the thesis provides theoretical knowledge about the physical properties of sound. Furthermore, the thesis includes a description of the experiment and the evaluation of the measured data. Finally, it discusses the technical solution to the problem itself and proposes a design modification that could solve the problem. However, this modification cannot yet be used in practice due to the uncertainty of the efficiency level.}}	% abstrakt

% Jména Autora a Vedoucího a Konzultanta
\newcommand{\Atituly}{}	% tituly před jménem Autora
\newcommand{\Ajmeno}{Jaroslav}	% křestní Autora			
\newcommand{\Aprijmeni}{Sýkora}	% příjmení Autora
\newcommand{\autor}{\Atituly\,\Ajmeno\ \Aprijmeni}	% celé jméno Autora

\newcommand{\VtitulyP}{doc. Ing.}	% tituly před jménem Vedoucího
\newcommand{\VtitulyZ}{Ph.D.}	% tituly za jménem Vedoucího
\newcommand{\Vjmeno}{Petr}	% křestní Vedoucího
\newcommand{\Vprijmeni}{Eret}	% příjmení Vedoucího
\newcommand{\vedouci}{\Vtituly\,\Vjmeno\ \Vprijmeni}	% celé jmnéno Vedoucího  		
\newcommand{\KtitulyP}{doc. Ing.}	% tituly před jménem Vedoucího
\newcommand{\KtitulyZ}{Ph.D.}	% tituly za jménem Vedoucího
\newcommand{\Kjmeno}{Michal}	% křestní Vedoucího
\newcommand{\Kprijmeni}{Hoznedl}	% příjmení Vedoucího
\newcommand{\konzultant}{\Ktituly\,\Kjmeno\ \Kprijmeni}	% celé jmnéno Vedoucího  

\newcommand{\rok}{2023/2024}	% akademický rok
\newcommand{\kde}{Plzni}	% místo podpisu prohlášení
\newcommand{\kdy}{24.05.2024}	% datum podpisu prohlášení
\newcommand{\Rokodevzdani}{2024}	% rok odevzdání práce

\newcommand{\Pcelkem}{67}	% celkový počet A4
\newcommand{\Ptext}{32}	% počet stran práce A4
\newcommand{\Pgraphic}{0}	% počet stran grafické práce v A4

\newcommand{\prohlaseni}{Předkládám tímto k posouzení a obhajobě bakalářskou práci zpracovanou na závěr studia na
Fakultě strojní Západočeské univerzity v Plzni.
Prohlašuji, že jsem tuto bakalářskou práci vypracoval samostatně, s použitím odborné literatury
a pramenů uvedených v seznamu, který je součástí této bakalářské práce.
}	% text prohlášení
\newcommand{\podekovani}{Děkuji vedoucímu práce doc. Ing. Petrovi Eretovi, Ph.D., za odborné vedení mé bakalářské práce. Dále bych rád poděkoval
konzultantovi doc. Ing. Michalovi Hoznedlovi, Ph.D., za vstřícnou spolupráci.

V neposlední řadě bych rád poděkoval také společnosti Doosan Škoda Power~s.~r.~o. za poskytnutí naměřených dat, bez kterých by bakalářská práce nemohla vzniknout.}	% text poděkování
