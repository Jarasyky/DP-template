\chapter*{Závěr}
\addcontentsline{toc}{chapter}{Závěr}

První část práce se zabývala definováním všech akustických veličin, které jsou dále v práci používány, například rychlostí zvuku v prostředí, frekvencí zvuku a tak dále. Byly zde popsány základní principy šíření zvuku v prostředí. Dalším důležitým bodem teoretické části práce byl popis akustických hladin a práce s logaritmickou škálou. v závěru teoretické části byla popsána frekvenční pásma použitá při vyhodnocování naměřených dat.
\par I přesto, že se práce zaměřuje na technické řešení problému, bylo nutné také popsat, jak lidské tělo reaguje na zvuk a proč je nutné dbát na hlučnost prostředí, ve kterém lidé pracují. 
\par Zásadní byl popis měřidel, která byla použita při experimentech. Při zkoumání problematiky provedla společnost Doosan Škoda Power hned několik experimentů, v práci jsou tedy zdůvodněny použité metody měření a stručně popsány varianty měření. Zobrazené jsou také výsledky měření z prvních třech variant, tyto výsledky byly zpracovány prostřednictvím Matlab scriptu.
\par Výsledky měření byly srovnány s výsledky získanými analytickou metodou. Tato metoda je vhodná pro předpovídání frekvenčních špiček pro daný otvor. Nelze však předpovídat velikost akustické hladiny vzniklého zvuku. Pro hlubší pochopení problému by bylo nejspíše nutné provést CFD a CAA simulace opřené o experiment.
\par Nakonec práce byla navržena konstrukční úprava spojky, která by měla snížit hlučnost. Tato úprava však zatím není v praxi použita, jelikož výsledek tohoto řešení není jistý a v případě nedostatečného útlumu by nebylo možné použít zaslepení děr \cite{bartel_possible_2021}. 