\chapter*{Úvod}\label{chap:uvod}
\addcontentsline{toc}{chapter}{Úvod}

Při prodeji, respektive nabízení turbíny, je potřeba garantovat zákazníkovi parametry tohoto zařízení. Tyto parametry se zpravidla liší a je na zákazníkovi, jak přísné požadavky bude mít. Například může jít o výkon turbíny, její účinnost, životnost, odolnost vůči zemětřesení a tak dále. 

Stále častěji je na výrobce turbín kladen nárok ohledně garance maximální hladiny akustického tlaku, kterou zařízení při provozu nesmí překročit. Hlavní důvod takového požadavku je ochrana zdraví pracovníků, kteří pracují v blízkosti zařízení. Takový požadavek vznášejí převážně vyspělé země, ve kterých je ochrana pracovníka na prvním místě. Schopnost garantovat maximální zvuk turbíny tedy může znamenat konkurenční výhodu.

Práce se bude nejprve zabývat fyzikální podstatou zvuku a odvozením základních vztahů. Následně budou popsány nesrovnalosti při porovnávání naměřených dat s vnímáním zvuku lidským organismem. S tím souvisí používaní filtrů, které problematiku berou v potaz.

Při analýze turbíny za provozu odhalila společnost Doosan Škoda Power několik zdrojů zvuku. Způsob měření je normalizován a popis tohoto měření je uveden níže v práci. Vyhodnocení dat ukázalo, že nejvýznamnějším zdrojem je rotující spojka mezi turbínou a generátorem. Proto se bakalářská práce zabývá právě touto spojkou.

Po obvodu a na čele spojky se nachází několik otvorů, které by mohly zvuk způsobovat. Závitové díry po obvodu slouží pro našroubování vývažků, válcové díry slouží pro pojištění šroubů proti povolení a díry na čele spojky slouží pro sešroubování dvou kotoučů spojky. Součástí práce bude také popsat, jak geometrie a umístění těchto děr ovlivňuje frekvenci a intenzitu zvuku. 

Tento problém lze řešit dvěma způsoby. Jako první řešení lze použít zvukopohltivé kryty, ty jsou ale drahé a nemají jistý výsledek. Druhou možností jsou záslepky pískajících otvorů. Ty jsou spolehlivější, ovšem jejich přidáním může vzniknout nevývažek. Proto je snaha pochopit přímou příčinu vzniku akustického tlaku. Na základě těchto poznatků je potřeba navrhnout takové řešení, aby k problému nedocházelo.

Pro správné pochopení problému je nutné nejdříve popsat fyzikální význam akustického tlaku. Přesněji to, jak se šíří, jakými způsoby ho můžeme měřit a také to, jak na problematiku pohlíží normy.  




